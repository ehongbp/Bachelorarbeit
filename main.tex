 \documentclass{common/nak}

%Deckblattkonfiguration
%Die Werte in den Klammern können entsprechend angepasst werden
\def\transferLeistungnnummer{69}
\def\matrikelNummer{42069}
\def\themenBeschreibung{Irgendein langweiliger shit der nichts mit Informatik zutun hat}
\def\studienGang{Angewandte Informatik}
\def\zenturie{A18a}

%General stuff
\usepackage{acronym}
\graphicspath{{images/}}
\usepackage{url}
\usepackage{outlines}
\usepackage{todonotes}
\usepackage{amsmath}
%%% --- The following two lines are what needs to be added --- %%%
\setcounter{biburllcpenalty}{7000}
\setcounter{biburlucpenalty}{8000}
\addbibresource{quellen.bib}

\begin{document}
\hspace{3cm}
%eigenes deckblatt
\begin{center}
    \includegraphics[width=0.7\textwidth]{images/nak_logo2.png}\\
     \huge { Bachelorarbeit \\[1em]
     \large{Analyse der Effektivität von Media-Kanälen zur Steigerung der Nachfrage:
     \\ Eine Untersuchung mit Marketing-Mix-Modell bei bonprix Handelsgesellschaft mbH
}}
\end{center}

\vspace*{\fill}


\noindent Vorgelegt von:\\
Enxi Hong \\
Zeisigring \\
24568 Kaltenkirchen\\
E-Mail: enxi.hong@nordakademie.de\\
\\
Matrikelnummer: 11952\\
Jahrgang: 2021\\
Studiengang: Angewandte Informatik\\
Datum: \today\\
Betreuender Dozent: Michael Fretschner\\

\newpage

%Seitenzahlen auf Römisch
\pagenumbering{Roman}

%Auskommentiert, weil expose
\iffalse
\includepdf{common/Transferleistung_Deckblatt_05_flattend}
\fi


\newpage
\section*{Sperrvermerk}

Die vorliegende Transferleistung beinhaltet interne vertrauliche Informationen der IT.UV Software GmbH. Die Weitergabe des Inhaltes der Arbeit und eventuell beiliegender Zeichnungen und Daten im Gesamten oder in Teilen ist grundsätzlich untersagt. Es dürfen keinerlei Kopien oder Abschriften - auch in digitaler Form - gefertigt werden. Ausnahmen bedürfen der schriftlichen Genehmigung der IT.UV Software GmbH.

\newpage

%Inhaltsverzeichnis
\tableofcontents
\newpage


\listoffigures
\addcontentsline{toc}{section}{Abbildungsverzeichnis}
%auskommentiert, wenn leer:

\newpage
\listoftables
\addcontentsline{toc}{section}{Tabellenverzeichnis}

\newpage
\addcontentsline{toc}{section}{Abkürzungsverzeichnis}
\addtocontents{toc}

\section*{Abkürzungsverzeichnis}
\begin{acronym}[Abkürzung] 
\acro{ac:Label}[Abkürzung]{Langform}
\acro{BoFu}{Bottom of the Funnel}  
\acro{dooh}{Digital-Out-of-Home}  
\acro{KPI}{Key Performance Indicators}  
\acro{LAD}{Least Absolute Deviations}  
\acro{MLE}{Maximum Likelihood Estimation}  
\acro{MLR}{Multiple Linear Regression}  
\acro{MMM}{Marketing-Mix-Modell}  
\acro{MoFu}{Middle of the Funnel}  
\acro{OMA}{Online-Marketing}  
\acro{ooh}{Out-of-Home}  
\acro{OLS}{Ordinary Least Squares}  
\acro{olv}{Online-Video}  
\acro{PR}{Public Relations}  
\acro{ROAS}{Return on Advertising Spend}  
\acro{ROI}{Return on Investment}  
\acro{RSS}{Residual Sum of Squares}  
\acro{ToFu}{Top of the Funnel}  
\acro{VIF}{Variance Inflation Factor}  
\end{acronym}
 

\newpage

\setcounter{page}{1}
%Seitenzahlen
\pagenumbering{arabic}
%Abbildungscounter sonst bei 2
\setcounter{figure}{0}

%Sectionen hinzufügen
\foreach \i in {00, 01, 02, 03, 04, 05, 06, 07, 08, 09, 10, ...,99} {%
    \edef\FileName{sections/file\i}%
    \IfFileExists{\FileName}{%
       \input{\FileName}%
    }
}


%Quellen
\newpage
\clearpage
\printbibliography

\newpage
%Anhang
\appendix
\section*{Anhang}
\addcontentsline{toc}{section}{Anhang}



\iffalse
\subsection*{Anhang 1}
\label{anhang:Ersterhebung}
\addcontentsline{toc}{subsection}{Anhang 1}
Anforderungen aus der Ersterhebung:
\begin{center}
    \includegraphics[height=0.88\textheight]{images/Metriken_ersterhebung_manuell.png}
\end{center}

\newpage
\subsection*{Anhang 2}
\label{anhang:gruppiert}
\addcontentsline{toc}{subsection}{Anhang 2}
Metriken nach Gruppierung und Konfliktlösung:
\begin{center}
    \includegraphics[width=1\textwidth]{images/Metriken_Gruppiert_manuell.png}
\end{center}
\fi

\newpage
\end{document}
