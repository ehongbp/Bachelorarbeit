 \documentclass{common/nak}

%General stuff
\usepackage{acronym}
\graphicspath{{images/}}
\usepackage{url}
\usepackage{outlines}
\usepackage{todonotes}
\usepackage{amsmath}
\usepackage{amsfonts}
\usepackage{listings}
\usepackage{xcolor}
\usepackage{booktabs}


\lstset{ 
    language=Python,
    basicstyle=\ttfamily\footnotesize,
    keywordstyle=\color{blue},
    commentstyle=\color{green},
    stringstyle=\color{red},
    frame=single, 
    breaklines=true,
    showstringspaces=false
}

\numberwithin{equation}{subsection}

%%% --- The following two lines are what needs to be added --- %%%
\setcounter{biburllcpenalty}{7000}
\setcounter{biburlucpenalty}{8000}
\addbibresource{quellen.bib}
\setlength {\marginparwidth }{2cm} 
\begin{document}
\hspace{3cm}
%eigenes deckblatt
\begin{center}
    \includegraphics[width=0.7\textwidth]{images/nak_logo2.png}\\
     \huge { Bachelorarbeit \\[1em]
     \large{Analyse der Effektivität von Media-Kanälen zur Steigerung der Nachfrage:
     \\ Eine Untersuchung mit Marketing-Mix-Modell bei bonprix Handelsgesellschaft mbH
}}
\end{center}

\vspace*{\fill}


\noindent 
Enxi Hong \\
E-Mail: enxi.hong@nordakademie.de\\
Matrikelnummer: 11952\\
Zenturie: A21a\\\\
Studiengang: Angewandte Informatik\\
Gutachter: Prof. Dr. Michael Fretschner\\
Co-Gutachter: Prof. Dr. Jörg Richard\\
Betrieblicher Betreuer: Tasmin Krugmann\\\\
Datum: \today

\newpage

%Seitenzahlen auf Römisch
\pagenumbering{Roman}
\newpage
\hspace{2cm}\\
\textbf{\Large Eidesstattliche Erklärung}\\\\
Mit meiner Unterschrift versichere ich, dass ich die hier vorliegende Arbeit 
selbständig, ohne fremde Hilfe und nur mit den angegebenen Hilfsmitteln verfasst 
habe und meine Angaben zu den verwendeten Quellen der Wahrheit entsprechen und 
vollständig sind. Alle Quellen, aus denen ich wörtlich oder sinngemäß übernommen 
habe, habe ich als solche gekennzeichnet.\\\\
Darüber hinaus versichere ich, dass ich sämtliche Teile der vorliegenden Arbeit, die 
unter Zuhilfenahme künstlicher Intelligenz (KI) generiert wurden, als solche gekennzeichnet habe und deren Entstehung in einer beigefügten Prozessdokumentation 
nachgewiesen habe. \\\\
Ich habe zur Kenntnis genommen, dass zuwiderlaufendes Verhalten als Täuschungsversuch gewertet wird und zu den in der geltenden Prüfungsverfahrensordnung 
genannten Konsequenzen führen wird.\\\\
 \\
Datum:\hspace{1cm}\today\\\\
Unterschrift:


\newpage
\section*{Sperrvermerk}

Die vorliegende Bachelorarbeit beinhaltet interne vertrauliche Informationen der bonprix Handelsgesellschaft mbH. Die Weitergabe des Inhaltes der Arbeit und eventuell beiliegender Zeichnungen und Daten im Gesamten oder in Teilen ist grundsätzlich untersagt. Es dürfen keinerlei Kopien oder Abschriften - auch in digitaler Form - gefertigt werden. Ausnahmen bedürfen der schriftlichen Genehmigung der bonprix Handelsgesellschaft mbH.

\newpage

%Inhaltsverzeichnis
\tableofcontents
\newpage


\listoffigures
\addcontentsline{toc}{section}{Abbildungsverzeichnis}
%auskommentiert, wenn leer:

\newpage
\listoftables
\addcontentsline{toc}{section}{Tabellenverzeichnis}

\newpage
\addcontentsline{toc}{section}{Abkürzungsverzeichnis}
\addtocontents{toc}

\section*{Abkürzungsverzeichnis}
\begin{acronym}[Abkürzung] 
\acro{ac:Label}[Abkürzung]{Langform}
\acro{BoFu}{Bottom of the Funnel}  
\acro{dooh}{Digital-Out-of-Home}  
\acro{KPI}{Key Performance Indicators}  
\acro{LAD}{Least Absolute Deviations}  
\acro{MLE}{Maximum Likelihood Estimation}  
\acro{MLR}{Multiple Linear Regression}  
\acro{MMM}{Marketing-Mix-Modell}  
\acro{MoFu}{Middle of the Funnel}  
\acro{OMA}{Online-Marketing}  
\acro{ooh}{Out-of-Home}  
\acro{OLS}{Ordinary Least Squares}  
\acro{olv}{Online-Video}  
\acro{PR}{Public Relations}  
\acro{ROAS}{Return on Advertising Spend}  
\acro{ROI}{Return on Investment}  
\acro{RSS}{Residual Sum of Squares}  
\acro{ToFu}{Top of the Funnel}  
\acro{VIF}{Variance Inflation Factor}  
\end{acronym}
 

\newpage

\setcounter{page}{1}
%Seitenzahlen
\pagenumbering{arabic}
%Abbildungscounter sonst bei 2
\setcounter{figure}{0}

%Sectionen hinzufügen
\foreach \i in { 01, 02, 03, 04, 05, 06, 07, 08, 99 } {%
    \edef\FileName{sections/file\i}%
    \IfFileExists{\FileName}{%
       \input{\FileName}%
    }
}


%Quellen
\newpage
\clearpage
\printbibliography

\newpage
%Anhang
\appendix
\section*{Anhang}
\label{anhang}
\addcontentsline{toc}{section}{Anhang}
\subsection*{Anhang 1: Zeitlicher Verlauf Nachfrage und Mediaausgabe}
\label{Anhang1:ZeitlicherVerlaufMitPywidgets}
\UseRawInputEncoding
\begin{lstlisting}[language=Python, linewidth=\textwidth]
    def plot_data(x_axis, y_axis):
    start_date_1 = y_axis.strftime("%Y-%m-%d")
    df_raw['date'] = pd.to_datetime(df_raw['date'])
    
    df_1 = df_raw[(df_raw['date'] >= "2022-01-01") & (df_raw['date'] <= "2025-01-01")].copy() 
    df_1['date'] = pd.to_datetime(df_1['date'])
    df_1['demand_percent'] = (df_1['demand'] / df_1['demand'].sum())
    df_1[f'{x_axis}_percent'] = (df_1[x_axis] / df_1[x_axis].sum())
    
    fig, ax1 = plt.subplots(figsize=(12, 8))
    
    line1, = ax1.plot(df_1['date'], df_1[f'{x_axis}_percent'], color='#FF69B4', label=f"{x_axis}")
    ax1.set_xlabel("Datum", fontsize=12)
    ax1.set_ylabel("Channel Costs", fontsize=12)
    
    ax2 = ax1.twinx()
    line2, = ax2.plot(df_1['date'], df_1['demand_percent'], color='#87CEFA', label="Demand Percent")
    ax2.set_ylabel("Demand Percent", fontsize=12)
    
    lines = [line1, line2]
    labels = [line.get_label() for line in lines]
    ax1.legend(lines, labels, loc="upper left")
    
    plt.title(f"Zeitlicher Verlauf der Werte für {x_axis}", fontsize=14)
    plt.xticks(rotation=45)
    plt.grid(True)
    plt.show()

x_axis_dropdown = widgets.Dropdown(
    options=["oma_costs", "organic_costs", "catalog_costs_total", "abs_discount_value", "sum_vkb", "c_and_a_costs", "h_and_m_costs", "aboutyou_costs", "zalando_costs","mail_newsletter", "mail_trigger", "push_newsletter", "push_trigger""tv_costs", "podcast_costs", "addressable_tv_costs", "radio_costs", "dooh_ooh_costs", "youtube_costs", "olv_costs", "display_media_costs", "social_media_costs", "media_print_costs"],
    description="X-Achse:",
)

y_axis_dropdown = widgets.DatePicker(
    description="Start Date",
    disabled=False,
    value=date(2022, 1, 1),
    min=date(2022, 1, 1),  
    max=date(2025, 1, 1),
)

interact(plot_data, x_axis=x_axis_dropdown, y_axis=y_axis_dropdown)
\end{lstlisting}
\newpage
\subsection*{Anhang 2: Kreisdiagramm Aufteilung der Mediaausgaben}
\label{Anhang2KreisdiagrammAufteilungDerMediaausgaben}
\begin{lstlisting}[language=Python, linewidth=\textwidth]
start_date_1 = "2022-01-01"
end_date_1 = "2024-11-30"
start_date_1 = pd.to_datetime(start_date_1)
end_date_1 = pd.to_datetime(end_date_1)

def plot_data(x_axis):
    start_date_1 = x_axis.strftime("%Y-%m-%d")
    df_1 = df_raw[(df_raw['date'] >= start_date_1) & (df_raw['date'] <= end_date_1)].copy() 
    df_1['date'] = pd.to_datetime(df_1['date'])
    channels = ["tv_costs", "podcast_costs", "addressable_tv_costs", "radio_costs", 
            "dooh_ooh_costs", "youtube_costs", "olv_costs", 
            "display_media_costs", "social_media_costs", "media_print_costs"]

    channel_sums = {channel: float(df_1[channel].sum()) for channel in channels}
    total_cost = sum(channel_sums.values())

    channel_proportions = {channel: (cost / total_cost) * 100 for channel, cost in channel_sums.items()}

    plt.figure(figsize=(8, 8))
    plt.pie(
        channel_sums.values(),
        labels=channel_sums.keys(),
        autopct='%1.1f%%',
        startangle=90
    )

    plt.title("Channel Cost Proportions", fontsize=14)
    plt.show()
    filename = 'ausgaben_media_pie.png'
    plt.savefig(filename, dpi=300, bbox_inches='tight')
    plt.close() 

x_axis_dropdown = widgets.DatePicker(
    description="Start Date",
    disabled=False,
    value=date(2022, 1, 1),
    min=date(2022, 1, 1),  
    max=date(2024, 10, 30),
)

interact(plot_data, x_axis=x_axis_dropdown)
\end{lstlisting}

\newpage
\end{document}
