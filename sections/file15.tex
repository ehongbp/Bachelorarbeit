\iffalse
\afterpage{%
\clearpage% Flush earlier floats (otherwise order might not be correct)
\begin{landscape}% Landscape page
%\begin{adjustbox}{angle=90}
\begin{table}[!ht] %sidwaystable
    \centering
    \resizebox{\textwidth}{!}{
    \begin{table}{|p{0.1\textwidth}|p{0.1\textwidth}|p{0.1\textwidth}|p{0.1\textwidth}|p{0.2\textwidth}|p{0.2\textwidth}|p{0.1\textwidth}|p{0.1\textwidth}|}
    \hline
        Metrik ID & Anforderungs ID & Person(en) & Metrikart & Beschreibung & Nutzungserklärung & Kontext & Wichtigkeitsgrad (maximal) \\ \hline
        1 & 1, (2), 14, 19, 49 & A, B, C & Servermetriken & alle Metriken zur Überwachung der physischen Eigenschaften eines Servers: CPU, RAM, … & Performanceanalyse & alle Umgebungen & 5 \\ \hline
        2 & 6 & A & Anwendungsmetriken & Statusinformationen zu Startzeit und Anwendungsversion & Erkennung von Versionsänderungen und Laufzeit & Entwicklungsumgebung intern & 4 \\ \hline
        3 & 2, 3, 4, 20, 50 & A, B, C & Anwendungsmetriken & alle Metriken zur Überwachung der Anwendung, welche den edoc.server darstellen: Threads, Garbage Collection, Heap Space, Java Version, … & Performanceanalyse, Engpasserkennung im Produktivsystem, Systemüberblick und Speicherleckerkennung & alle Umgebungen & 5 \\ \hline
        4 & 8, 39, 42, 48 & A, C, H & Anwendungsmetrik & Der Status der jeweiligen Dokumentklassenkonnektoren (aktiv/inaktiv) & Statusüberwachung & alle Umgebungen & 5 \\ \hline
        5 & 5, 39, 42, 48 & A, C, H & Anwendungsmetrik & Der Status der jeweiligen Dokumentklassen (aktiv/inaktiv) & Statusüberwachung & alle Umgebungen & 5 \\ \hline
        6 & 7 & A & fachliche Metriken & Übersicht über alle Aufträge des edoc.servers, Status: Angenommen, Fehlerhaft, Archiviert, Geparkt, Laufend, Ausgeführt & ~ & Entwicklungsumgebung intern & 4 \\ \hline
        7 & 10 & A & Anwendungsmetrik & Anzahl der HTTP-Anfragen je nach Pfad und Responsecodes & Restschnittstellenüberwachung & Entwicklungsumgebung & 3 \\ \hline
        8 & 5 & A & Anwendungsmetrik & Anzahl der Threads pro Dokumentklasse & Statusüberwachung & Entwicklungsumgebung & 4 \\ \hline
        9 & 5, 16, 23, 45 & A, B, C, G & fachliche Metrik & Anzahl der Aufträge, Status: Eingang & Performanceanalyse, Ressourcenmanagement & alle Umgebungen & 4 \\ \hline
        10 & 5, 46 & A, C & fachliche Metrik & Anzahl der Aufträge, Status: In Arbeit & Performanceanalyse & alle Umgebungen & 4 \\ \hline
        11 & 5, 11, 12, 22, 30, 32, 34, 38, 41, 47 & A, B, C, E, F, H & fachliche Metrik & Anzahl der Aufträge, Status: Ausgeführt & Performanceanalyse, Ressourcenmanagement & alle Umgebungen & 5 \\ \hline
        12 & 21 & B & fachliche Metrik & Anzahl der angebotenen Dokumentklassen & Statusüberwachung & Produktivumgebung BGW & 1 \\ \hline
        13 & 9 & A & Anwendungsmetrik & Anzahl der Fehler pro Dokumentklassenkonnektor & Statusüberwachung & Entwicklungsumgebung intern & 2 \\ \hline
        14 & 5, 18, 24, 29, 31, 33, 35, 40 & A, B, C, D, G, E, H & fachliche Metriken & Anzahl der Aufträge, Status: Fehlerhaft & Fehleranalyse, Statusüberwachung,  & alle Umgebungen & 5 \\ \hline
        15 & 10, 25, 51 & A, C, D & Anwendungsmetrik & Anzahl der HTTP-Anfragen & Performanceanalyse von Dokumentanfragen & alle Umgebungen & 3 \\ \hline
        16 & 10, 28 & A, D & Anwendungsmetrik & Anzahl der erfolglosen HTTP-Anfragen & Fehleranalyse, Statusüberwachung & alle Umgebungen & 5 \\ \hline
        17 & 52 & C & Anwendungsmetrik & Dauer der jeweiligen HTTP-Anfragen & Performanceanalyse von Dokumentanfragen & alle Umgebungen & 3 \\ \hline
        18 & 26, 27 & D & fachliche Metrik & Anzahl der unterschiedlichen Nutzer pro Bezirksverwaltung, die bestimmte HTTP-Anfragen stellen & Performanceanalyse von Dokumentanfragen & Produktivumgebung BGW & 3 \\ \hline
    \end{table}
    }
    \caption{Metriken nach Gruppierung der Ersterhebung aus \protect\hyperlink{Anhang 1}{anhang1}}
    \label{tab:metrikenNachGruppierung}
\end{table}
%\end{adjustbox}

\end{landscape}
\clearpage% Flush page
}

\begin{landscape} % Rotate the pages
\setlength\LTleft{0pt} % Set left margin for longtable
\setlength\LTright{0pt} % Set right margin for longtable
    \begin{longtable}{|p{0.1\textwidth}|p{0.1\textwidth}|p{0.1\textwidth}|p{0.1\textwidth}|p{0.2\textwidth}|p{0.2\textwidth}|p{0.1\textwidth}|p{0.1\textwidth}|}
    \hline
    Metrik ID & Anforderungs ID & Person(en) & Metrikart & Beschreibung & Nutzungserklärung & Kontext & Wichtigkeitsgrad (maximal) \\ \hline
    \endfirsthead
    \hline
    Metrik ID & Anforderungs ID & Person(en) & Metrikart & Beschreibung & Nutzungserklärung & Kontext & Wichtigkeitsgrad (maximal) \\ \hline
    \endhead
    \hline
    \endfoot
    \hline
    \endlastfoot
    \hline
        Metrik ID & Anforderungs ID & Person(en) & Metrikart & Beschreibung & Nutzungserklärung & Kontext & Wichtigkeitsgrad (maximal) \\ \hline
        1 & 1, (2), 14, 19, 49 & A, B, C & Servermetriken & alle Metriken zur Überwachung der physischen Eigenschaften eines Servers: CPU, RAM, … & Performanceanalyse & alle Umgebungen & 5 \\ \hline
        2 & 6 & A & Anwendungsmetriken & Statusinformationen zu Startzeit und Anwendungsversion & Erkennung von Versionsänderungen und Laufzeit & Entwicklungsumgebung intern & 4 \\ \hline
        3 & 2, 3, 4, 20, 50 & A, B, C & Anwendungsmetriken & alle Metriken zur Überwachung der Anwendung, welche den edoc.server darstellen: Threads, Garbage Collection, Heap Space, Java Version, … & Performanceanalyse, Engpasserkennung im Produktivsystem, Systemüberblick und Speicherleckerkennung & alle Umgebungen & 5 \\ \hline
        4 & 8, 39, 42, 48 & A, C, H & Anwendungsmetrik & Der Status der jeweiligen Dokumentklassenkonnektoren (aktiv/inaktiv) & Statusüberwachung & alle Umgebungen & 5 \\ \hline
        5 & 5, 39, 42, 48 & A, C, H & Anwendungsmetrik & Der Status der jeweiligen Dokumentklassen (aktiv/inaktiv) & Statusüberwachung & alle Umgebungen & 5 \\ \hline
        6 & 7 & A & fachliche Metriken & Übersicht über alle Aufträge des edoc.servers, Status: Angenommen, Fehlerhaft, Archiviert, Geparkt, Laufend, Ausgeführt & ~ & Entwicklungsumgebung intern & 4 \\ \hline
        7 & 10 & A & Anwendungsmetrik & Anzahl der HTTP-Anfragen je nach Pfad und Responsecodes & Restschnittstellenüberwachung & Entwicklungsumgebung & 3 \\ \hline
        8 & 5 & A & Anwendungsmetrik & Anzahl der Threads pro Dokumentklasse & Statusüberwachung & Entwicklungsumgebung & 4 \\ \hline
        9 & 5, 16, 23, 45 & A, B, C, G & fachliche Metrik & Anzahl der Aufträge, Status: Eingang & Performanceanalyse, Ressourcenmanagement & alle Umgebungen & 4 \\ \hline
        10 & 5, 46 & A, C & fachliche Metrik & Anzahl der Aufträge, Status: In Arbeit & Performanceanalyse & alle Umgebungen & 4 \\ \hline
        11 & 5, 11, 12, 22, 30, 32, 34, 38, 41, 47 & A, B, C, E, F, H & fachliche Metrik & Anzahl der Aufträge, Status: Ausgeführt & Performanceanalyse, Ressourcenmanagement & alle Umgebungen & 5 \\ \hline
        12 & 21 & B & fachliche Metrik & Anzahl der angebotenen Dokumentklassen & Statusüberwachung & Produktivumgebung BGW & 1 \\ \hline
        13 & 9 & A & Anwendungsmetrik & Anzahl der Fehler pro Dokumentklassenkonnektor & Statusüberwachung & Entwicklungsumgebung intern & 2 \\ \hline
        14 & 5, 18, 24, 29, 31, 33, 35, 40 & A, B, C, D, G, E, H & fachliche Metriken & Anzahl der Aufträge, Status: Fehlerhaft & Fehleranalyse, Statusüberwachung,  & alle Umgebungen & 5 \\ \hline
        15 & 10, 25, 51 & A, C, D & Anwendungsmetrik & Anzahl der HTTP-Anfragen & Performanceanalyse von Dokumentanfragen & alle Umgebungen & 3 \\ \hline
        16 & 10, 28 & A, D & Anwendungsmetrik & Anzahl der erfolglosen HTTP-Anfragen & Fehleranalyse, Statusüberwachung & alle Umgebungen & 5 \\ \hline
        17 & 52 & C & Anwendungsmetrik & Dauer der jeweiligen HTTP-Anfragen & Performanceanalyse von Dokumentanfragen & alle Umgebungen & 3 \\ \hline
        18 & 26, 27 & D & fachliche Metrik & Anzahl der unterschiedlichen Nutzer pro Bezirksverwaltung, die bestimmte HTTP-Anfragen stellen & Performanceanalyse von Dokumentanfragen & Produktivumgebung BGW & 3 \\ \hline
    \caption{Metriken nach Gruppierung der Ersterhebung aus \protect\hyperlink{Anhang 1}{anhang1}}
    \label{tab:metrikenNachGruppierung}
    \end{longtable}

\end{landscape}

\fi