\iffalse

\begin{table}[!ht]
    \centering
    \caption{Zugangspunkte von mit \anf{mogena.uv} entwickelten Webapplikationen}
    \label{zugangspunkteTabelle}
    \resizebox{0.8\columnwidth}{!}{\begin{tabular}{|L{3cm}|L{12cm}|}
    \hline
        Name & Beschreibung \\ \hline
        HTTPS Port & Alle Anwender haben die Möglichkeit die Webapplikation unter dem HTTPS Port zu erreichen. \\ \hline
        Login Seite & Die Seite, welche alle Anwender zuerst angezeigt bekommen. Alle Aufrufe werden auf diese Seite umgeleitet, sofern keine Anmeldung stattgefunden hat.\\ \hline
        iFrame-einbettende Seite & Eine modellierte Seite, welche die Implementation des iFrames beinhaltet. \\ \hline
        Vaadin Frontend Engine & Der Browser-seitige Client von Applikationen, welche mit Vaadin erstellt wurden.\\ \hline
        DOM Listener auf Events & Alle mit Vaadin entwickelten Applikationen verwenden DOM Events, um Updates für die Komponenten zu senden, welche auf der Webseite verwendet werden. \\ \hline
        Nachrichten zischen den iFrame und dem Window & Es besteht die Möglichkeit Nachrichten, welche zwischen den Fenstern geschickt werden abzufangen und einzusehen. \\ \hline
    \end{tabular}}
\end{table}

\begin{table}[!ht]
    \centering
    \caption{Liste der Bedrohungen nach dem Drei-Schichten-Modell mit Erweiterungen um Eigenschaften}
    \label{bedrohungenTabelle}
    \resizebox{\columnwidth}{!}{\begin{tabular}{|L{1cm}|L{5cm}|L{5cm}|L{5cm}|L{5cm}|}
    \hline
        Nr. & Schwachstelle & theoretisches Beispiel & Angriffsklasse & Auswirkung auf Geschäftstätigkeit \\ \hline
        1 & Injektion von Code (über Events) & Triggern von anderem Verhalten auf der Webseite. Veränderung von Daten ist nicht möglich. & Code Injektion & Verlust der Integrität \\ \hline
        2 & Verarbeitung von fremden Events durch den Public-Listener & Es ist möglich Daten und Anweisungen über den Listener zu verarbeiten. Unrechtmäßige Veränderungen werden in der Regel abgefangen. & Tampering und Information Disclosure & Verlust der Vertraulichkeit und Verlust der Authentizität \\ \hline
        3 & Abfangen der unverschlüsselten Kommunikation zwischen den Browsern  & Abfangen einer Nachricht. Verändern des Inhaltes und erneutes Versenden. Noch keine Verschlüsselung und Identifizierung der Datenpakete. & Information Disclosure & Verlust der Vertraulichkeit und Verlust der Zurechenbarkeit \\ \hline
        4 & Abfangen der unverschlüsselten JSON-Dateien & Aus den JSON-Dateien können Nutzerdaten und sensible Daten entnommen werden.  & Information Disclosure & Verlust der Vertraulichkeit \\ \hline
        6 & Clickjacking & Die eingebundene Webseite wird auf einer nachgebauten Webseite eingebunden und die Eingaben werden abgehört (ähnlich dem Vorgehen von spear phishing).  & InformationDiscolsure Elevation of privilege & Verlust der Vertraulichkeit \\ \hline
    \end{tabular}}
\end{table}
\noindent

\begin{table}[!ht]
    \centering
    \caption{Wahrscheinlichkeitsskala}
    \label{wahrscheinlichkeitskala}
    \resizebox{\columnwidth}{!}{\begin{tabular}{|L{4cm}|L{3cm}|L{3cm}|L{3cm}|L{3 cm}|}
    \hline
        ITUV Punktwert & 1  & 2 & 3 & 5 \\ \hline
         & unwahrscheinlich & möglich & wahrscheinlich & sehr wahrscheinlich \\ \hline
        Annual Rate of Occurance (kurz: ARO) & 1x / 10 Jahre & 1x / Jahr &  1x / Monat &  mehr als 1x / Woche \\\hline
    \end{tabular}}
\end{table}
\noindent


\begin{table}[!ht]
    \centering
    \caption{Bewertung der Bedrohungen}
    \label{schaden+wahrscheinlichkeitTabelle}
    \resizebox{0.7\columnwidth}{!}{\begin{tabular}{|L{1cm}|L{3cm}|L{3cm}|L{3cm}|L{3cm}|}
    \hline
        Nr. & Schwachstelle  & Eintrittswahr-scheinlichkeit & potenzieller Schaden & Gesamtwertung \\ \hline
        1 & Injektion von Code (über Events) & 1 & 1 & 1 \\ \hline
        2 & Verarbeitung von fremden Events durch den Public-Listener & 1 & 3 & 3 \\ \hline
        3 & Abfangen der unverschlüsselten Kommunikation zwischen den Browsern  & 5 & 4 & 20 \\ \hline
        4 & Abfangen der unverschlüsselten JSON-Dateien & 5 & 5 & 25 \\ \hline
        5 & Clickjacking & 2 & 5 & 10 \\ \hline
    \end{tabular}}
\end{table}



\fullref{aufrufPostMessage}
\begin{codeBlock}{js}{Aufruf von postMessage im Konnektor}{aufrufPostMessage}
    this.iframe.contentWindow.postMessage('messagePort', '*', [this.channel.port2]); 
\end{codeBlock}

\begin{center}
\captionsetup{type=figure}
\includegraphics[width=0.7\textwidth]{images/Die-vier-Prinzipien-der-Barrierefreiheit.png}
\captionof{figure}{Die-vier-Prinzipien-der-Barrierefreiheit \cite[Abbildung 1]{WCAG-Artikel}}
\label{BF-Prinzipien-Bild}
\end{center}

\begin{center}
\captionsetup{type=figure}
\includegraphics[width=0.9\textwidth]{images/qualitaetssicherung_ituv.png}
\captionof{figure}{Qualitätssicherungsstrategien der ITUV}
\label{qualitaetssicherung_ituv}
\end{center}
\begin{figure}
    \centering
    \includegraphics[width=0.2\textwidth]{images/ELPO-Klassifikation-1.png}
    \caption{Klassifikation des Dokuments anhand eines Regelbaumes. \protect\\Quelle: eigene Darstellung} % \protect\\ wenn Zeilenumbruch
    \label{fig:ELPO-Klassifikation}
\end{figure}

%anpassung levellung 
\renewcommand{\labelenumii}{\arabic{enumi}.\arabic{enumii}}
\renewcommand{\labelenumiii}{\arabic{enumi}.\arabic{enumii}.\arabic{enumiii}}
\renewcommand{\labelenumiv}{\arabic{enumi}.\arabic{enumii}.\arabic{enumiii}.\arabic{enumiv}}

\paragraph{Richtlinien des BITV-Konformitätskonzept der ITUV}
\label{richtlinien}
\begin{enumerate}
    \item Wahrnehmbarkeit
    \begin{enumerate} 
        \item Textalternativen: Stellen Sie Textalternativen für alle Nicht-Text-Inhalte zur Verfügung, sodass diese in andere vom Benutzer benötigte Formen geändert werden können, wie zum Beispiel Großschrift, Braille, Symbole oder einfachere Sprache.
        \item Zeitbasierte Medien: Stellen Sie Alternativen für zeitbasierte Medien zur Verfügung oder fügen Sie Alternativtexte, Untertitel oder Audio-Deskriptionen hinzu.
        \item Anpassbar: Erstellen Sie Inhalte, die auf verschiedene Arten dargestellt werden können (z.B. einfacheres Layout), ohne dass Informationen oder Struktur verloren gehen.
        \item Unterscheidbar: Machen Sie es Benutzern leichter, Inhalt zu sehen und zu hören, einschließlich der Trennung von Vorder- und Hintergrund.
    \end{enumerate}
    \item Bedienbarkeit
    \begin{enumerate}
        \item Per Tastatur zugänglich: Sorgen Sie dafür, dass alle Funktionalitäten per Tastatur zugänglich sind.
        \item Ausreichend Zeit: Geben Sie den Benutzern ausreichend Zeit, Inhalte zu lesen und zu benutzen.
        \item Anfälle: Gestalten Sie Inhalte nicht auf Arten, von denen bekannt ist, dass sie zu Anfällen führen.
        \item Navigierbar: Stellen Sie Mittel zur Verfügung, um Benutzer dabei zu unterstützen zu navigieren, Inhalte zu finden und zu bestimmen, wo sie sich befinden.
        \item Eingabemodalitäten: Erleichtern Sie Benutzern die Bedienung von Funktionen über unterschiedliche Eingabemöglichkeiten abseits der Tastatur.
    \end{enumerate}
    \item Verständlichkeit
    \begin{enumerate} 
        \item Lesbar: Machen Sie Inhalt lesbar und verständlich.
        \item Vorhersehbar: Sorgen Sie dafür, dass Webseiten vorhersehbar aussehen und funktionieren.
        \item Hilfestellung bei der Eingabe: Helfen Sie den Benutzern dabei, Fehler zu vermeiden und zu korrigieren.
    \end{enumerate}
    \item Robustheit
    \begin{enumerate}
        \item Kompatibel: Maximieren Sie die Kompatibilität mit aktuellen und zukünftigen Benutzeragenten, einschließlich assistierender Techniken
    \end{enumerate}
\end{enumerate}

\fi
