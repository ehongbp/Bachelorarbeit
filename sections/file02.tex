\newpage
\section{Geschäftliche Grundlagen (ca. 12 Seiten)}
Das \ac{MMM} hat einen langen Weg zurückgelegt seit seiner Entstehung in 1950. \ac{MMM} entwickelte sich mit der Verbreitung des digitalen Marketings weiter. Später wurde \ac{MMM} durch die Fragmentierung der Medienkanäle und den Signalverlust beeinflusst, die durch stetig wechselnde Datenschutz- und regulatorische Rahmenbedingungen verursacht sind\cite{MMMdef}.
\subsection{Einführung in das Marketing-Mix-Modell}
Das Marketing-Mix-Modell misst den Einfluss der Marketing-Aktivitäten auf die Nachfrage. Ein Marketing-Mix-Modell hilft einer Firma dabei, zukünftige Ausgaben und ihre Kapitalrendite (Engl. ROI, Return On Invest) zu maximieren. Dabei misst das \ac{MMM} alle möglichen Marketingeffekte und deckt die Marketinginvestitionen auf, die ein langfristiges Ertragswachstum erzielen. Das sind die Variablen, die Marketing-Managers kontrollieren, um den Verkauf in der Firma zu beeinflussen. \\\\Das Fachwort \anf{Mix} in \ac{MMM} referiert die klassischen \anf{4Ps} im Marketing: Product, Price, Place and Promotion (Engl. Produkt, Preis, Ort und Verkaufsförderung)\cite{akinkunmi2018data}. \\\\
\begin{tikzpicture}[node distance=5.5cm, auto, centered]

\node (target) [ellipse, draw, text centered, minimum width=4cm, minimum height=2cm, text width=3.5cm] {ZIELMARKT};
\node (product) [ellipse, draw, above left of=target, text centered, minimum width=3cm, minimum height=2cm, text width=4cm] {Produkt \\ Was soll produziert und verkauft werden};
\node (place) [ellipse, draw, above right of=target, text centered, minimum width=3cm, minimum height=2cm, text width=3.5cm] {Ort \\ Wo soll das Produkt verkauft werden};
\node (promotion) [ellipse, draw, below left of=target, text centered, minimum width=3cm, minimum height=2cm, text width=6cm] {Werbung \\ Werbung, persönliche Verkaufsförderung, Verkaufsaktionen und Öffentlichkeitsarbeit};
\node (price) [ellipse, draw, below right of=target, text centered, minimum width=3cm, minimum height=2cm, text width=3.5cm] {Preis \\ Wie viel soll das Produkt kosten};

\draw[->] (product) -- (target);
\draw[->] (place) -- (target);
\draw[->] (promotion) -- (target);
\draw[->] (price) -- (target);


\node[below=1cm of promotion, text centered] {Abbildung 1: Marketing-Mix-Modell (\ac{MMM}) mit den vier P's \cite{akinkunmi2018data}};
\label{fig:mmm}

\end{tikzpicture}
\\\\Produkt steht für die Produkte der Firma und warum sie aus den Konkurrenzprodukten ausstechen. Darunter werden beispielsweise hohe Qualität, äußerliche Erscheinung und Wartung verstanden. Werbung bedeutet die Darstellung der Produkte seitens der Firma mithilfe von Werbungen, Gutscheine, Handelsmessen etc. Der Preis umfasst die Preisstrategien, die in Verbindung zur Produktpalette der Firma stehen. Diese hängen von dem Produkt-Lebenszyklus, den Produkteigenschaften, dem wahrgenommenen Nutzen und der Konkurrenz ab. Ort beschreibt den Zustellungsbereich der Produkte quantifiziert durch die Variablen wie Distribution, Verfügbarkeit und Bequemlichkeit\cite{akinkunmi2018data}. \\\\
Die Entwicklung des \ac{MMM}s wurde geprägt durch die Suche nach der optimalen Allokation der Marketingausgaben. Darunter werden Fragen gestellt wie: Was für eine Kombination von Marketing-Mix-Variablen maximiert die Ausgaben wie Unternehmensumsatz oder Marktanteil am Gewinn? Wie reagieren diese auf die Ausgaben dieser Marketing-Mix-Variablen? Darauffolgend haben Forscher ein wirtschaftliches Modell entwickelt, das diese Problematik behandelt. Das \ac{MMM} verwendet vergangene Daten, um valide Konklusion zu erstellen und bessere Marketingstrategien für die Zukunft zu entwickeln.  
\subsection{Media-Kanäle Definition}
Werbeausgaben sind Ausgaben eines Unternehmen, um auf einer Werbefläche ihre Produkte oder Dienstleistungen zu bewerben. Diese werden in zwei Hauptkanäle, traditionelle Werbung und digitale Werbung, umfasst. Die traditioneller Werbung wird definiert durch die Above-the-Line-Medien, die die Werbebotschaften einem breiten Publikum übermitteln. Die Above-the-Line-Medien umfassen Massenmedien wie traditionelles Fernsehen, traditionelles Radio, gedruckte Zeitungen, Magazine und Out-Of-Home-Werbeformate. Zur digitalen Werbung gehören Medien, die über das Internet ihre Werbebotschaften an Internetnutzern übermitteln. Darunter fallen digitale Out-Of-Home-Werbungen, digitale Werbebanner, digitale Audiowerbung, digitale Kleinanzeigenwerbungen, Werbungen in digitalen Videos und Suchmaschinen sowie Influencer-Werbung\cite{statista_werbung}. Werbeausgaben der Media-Kanäle beziehen sich auf Werbeausgaben für Werbefläche in einem Medium. Da bonprix bestimmte Media-Kanäle intern definiert hat, werden nur diese in diesem Kapital ausführlich beschrieben. \\\\

\subsection{Media-Kanäle in der Praxis}
Mit wachsenden Kosten in Media-Kanälen gewinnt die Aufteilung der Kosten pro Kanal an Bedeutung. \\\\
Die Versicherung Generali hat in 2022 in ihrem \ac{MMM} herausgefunden, dass YouTube im Vergleich zu anderen Media-Kanälen überproportional stark auf die Steigerung von Markenbekanntheit, Markenpräferenz und Versicherungsabschlüssen einwirkt\cite{DasZusammenspielKundenloyalität2022}. Die Nutzer sollen während des Lockdowns gelernt zu haben, YouTube als Erklärinhalte zu nutzen. So hat die YouTube-Werbung alle gemessenen Loyalitätsmetriken von der Versicherung positiv beeinflusst. 44\% der Nutzer, die solche Videos gesehen haben, kategorisieren die Versicherung als ihre erste Wahl. 15\% mehr Nutzer zogen einen Anbieterwechsel in Betracht und 6\% mehr Nutzer sprachen für eine Weiterempfehlung aus\cite{DasZusammenspielKundenloyalität2022}. \\\\
Ein weiterer Bericht thematisiert den Wandel der Media-Kanäle. Eine von Nielsen im Auftrag von Google durchgeführte Metaanalyse zeigt, dass YouTube ein Effektivitätstreiber ist und das lineare TV beim \ac{ROI} (Engl. Rentabilität der Investition) übertrifft. Das Nutzungsverhalten für Media wird fragmentierter und das TV verliert an Bedeutung. Bei Personen unter 40 beträgt die Nutzung von linearem TV inzwischen weniger als 50\% der gesamten Bewegtbildnutzung\cite{237097}.\\\\
Die Ergebnisse zur Abverkaufswirkung aus 144 \ac{MMM} wurden untersucht. Dabei zeigt sich, dass vier von fünf Unternehmen von höheren YouTube-Ausgaben in ihrem Marketing-Mix profitieren würden, um den Bewegtbild-ROI für TV und YouTube zu maximieren\cite{237097}. \\\\
Auch Offline-Medien haben eine Auswirkung auf das Kaufverhalten. Die zentrale Marktforschungsfirma von Deutschlands größter Mediaagenturgruppe, M-Science, brachte im Mai 2019 das Tool \anf{Crossoptimiser} zur Welt. Das Tool behandelt das Problem der Wirkungsforschung, dass manche Einflussfaktoren auf das Werbeergebnis übersehen werden. Die gängigen Modelle ordnen messbare Erfolge allein Online-Kanälen zu. Beispielsweise werden die Bekanntheit, generierte Adressen und Kaufabschlüsse dem letzten Werbeklick der Nutzeraktivität im Netz honoriert. Dieser Schluss übersieht die Wirkungsbeiträge, die Offline-Medien zu einem früheren Zeitpunkt im Wahrnehmungs- und Verkaufsprozess leisten. Vergleichbar mit dem Marketingtrichter, Nutzer finden und bestellen online Produkte, weil sie früher eine Print- oder TV-Werbung gesehen haben. Karin Immenroth, die Geschäftsführerin und zugleich Marktforscherin der Group M, zieht ihre ersten Schlussfolgerungen nach einem Jahr Vorarbeit und hunderten durchgerechneten Kampagnen. Ihre Schlussfolgerungen besagen, gerade bei Kunden mit einem hohen digitalen Werbeanteil wird der Wirkungsbeitrag von Offline oft stark unterschätzt. Je weiter sie im Marketingtrichter zurückforschen, desto stärker zeigt sich der Einfluss von Offline-Medien. Besonders bei Marken mit noch relativ niedrigen Awareness-Werten \cite[S. 4]{20190411492848}.
\\\\
Eine optimale Verteilung der Media-Kanäle kann dabei das maximale Potenzial jedes Kanals ausschöpfen. Denn Unternehmen, die in wirtschaftlich herausfordernden Zeiten mit geprüfter und gesteigerter Effizienz der Maßnahmen in das Marketing investieren, sind überdurchschnittlich erfolgreich\cite[S. ]{237097}. \\\\

\begin{figure}[h]
    \centering
    \includegraphics[width=0.75\linewidth]{images/werbeempfaenglichkeit.png}
    \caption{Entwicklung der Werbeempfänglichkeit von Verbrauchern nach Medienkanal in den Jahren 2012 und 2024 \cite{kantar_werbeempfaenglichkeit}}
    \label{fig:enter-label}
\end{figure}
%\subsection{Einfluss der Markenbekanntheit}
%\todo{hier sagen, was für einen Einfluss Markenbekanntheit auf den Umsatz/die Nachfrage hat}
\newpage
\subsection{Marketing-Mix-Modell bei bonprix} 
\todo{Ist Situation bei bonprix. Marketing-Mix-Modell und ROAS, Media-Mix und ROAS aktuell}

\begin{figure} [h]
    \centering
    \includegraphics[width=0.85\linewidth]{mmm.png}
    \caption{Marketing-Mix-Modell bei bonprix, eigene Darstellung}
    \label{fig:mmmbonprix}
\end{figure}

\newpage
\subsection{Media-Kanäle bei bonprix} 
Im \ac{MMM} von bonprix nehmen die Media-Ausgaben nur einen kleinen Anteil der Gesamtkosten ein. Der Kostenanteil von Media wird im Folgenden mit etwa zweijährigen, überlappenden Zeitfenstern betrachtet: Zwischen dem ersten Quartal 2020 und dem vierten Quartal 2021 betrugen die Media-Ausgaben 0,5\%. Sie stiegen im nächsten Zeitraum, vom dritten Quartal 2020 bis zum zweiten Quartal 2022, auf 1,5\%. Zum Schluss erhöhten sich die Media-Ausgaben vom ersten Quartal 2021 bis zum vierten Quartal 2022 auf 1,6\%. Um die Media-Ausgaben mit den Kosten anderer Kanälen zu vergleichen, werden alle Kosten als Anteil in Form von einer Tabelle in \autoref{tab:mmmausgaben} aufgelistet. \\\\
Da die Media-Ausgaben am Anfang der \ac{MMM}-Entwicklung sehr gering ist, wurde die Effizienz-Analyse ihrer Unterkanäle vernachlässigt.  
\begin{table}[h!]
\centering
\renewcommand{\arraystretch}{1.3}
\setlength{\tabcolsep}{10pt}
\begin{tabular}{|l|c|c|c|}
\hline
\textbf{Einflussfaktor} & \textbf{Q1-20/Q4-21} & \textbf{Q3-20/Q2-22} & \textbf{Q1-21/Q4-22} \\ \hline
OMA                    & 36,5\%               & 29,8\%               & 18,0\%               \\ \hline
Katalog                & 8,7\%                & 8,4\%                & 7,2\%                \\ \hline
Media                  & 0,5\%                & 1,5\%                & 1,6\%                \\ \hline
Mail                   & 2,5\%                & 4,2\%                & 3,0\%                \\ \hline
Rabatt                 & 10,1\%               & 13,1\%               & 15,4\%               \\ \hline
VKB                    & 6,6\%                & 4,7\%                & 4,7\%                \\ \hline
Externe Faktoren       & 14,2\%               & 19,7\%               & 27,6\%               \\ \hline
Konstante              & 21,0\%               & 18,6\%               & 22,5\%               \\ \hline
\end{tabular}
\caption{\ac{MMM}-Ausgaben-Tabelle mit überlappenden Zeitfenstern von jeweils ungefähr zwei Jahren}
\label{tab:mmmausgaben}
\end{table}