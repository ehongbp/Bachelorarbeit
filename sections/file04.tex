\newpage
\section{Umsetzung (ca. 10 Seiten)}
Nachdem theoretischen Grundlagen im letzten Kapitel vorgestellt werden, wird in diesem Kapitel die Umsetzung beschrieben. Dieser Kapitel wird nach einem geordneten Vorgehen strukturiert. Zuerst werden verfügbare Daten vorgestellt, analysiert und sortiert. In diesem Schritt wird entschieden, welche Daten die Zielvariable Nachfrage beeinflussen und somit mit in das Modell eingegeben werden. Deskriptive Analyse unterstützt dabei bei der Visualisierung der Daten und verleiht mehr Verständnis für den Ablauf der Media-Kanälen. Darauffolgend werden die Daten in das OLS Modell eingesetzt. Die Ergebnisse von dem Modell werden analyisiert.
\subsection{Daten vorstellen}
Zuerst werden die Daten für das \ac{MMM} ausgewertet. Für das \ac{MMM} bei bonprix werden Daten ab dem 01.01.2022 vorbereitet. Da Media ein Unterkanal des \ac{MMM}s ist, werden die \ac{MMM}-Daten auch für Media verwendet. \\\\

\subsection{Daten auswerten}
\subsection{Einsetzung des Marketing-Mix-Modells}
\subsection{Modell-Ergebnis}
