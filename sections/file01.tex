\newpage
\section{Einführung}
\label{Einführung}
\subsection{Motivation}
Die Motivation dieser Arbeit ergibt sich aus der aktuellen Relevanz des Media-Mix und der zunehmenden Bedeutung von datenbasierten Marketing-Analysen. Medienverantwortliche stehen vor der Herausforderung, möglichst viele Menschen zu erreichen und den Traffic zu steigern. Gleichzeitig ist es für sie von Interesse zu verstehen, wie effektiv sie das Publikum erreichen und welchen langfristigen Einfluss einzelne Media-Kanäle auf die Nachfrage haben. Dabei dient der Nachfrageeffekt der Media-Kanäle als Grundlage für fundierte Entscheidungen.\par
Eine weitere Motivation besteht darin, die optimale Verteilung der Mediaausgaben zu identifizieren, um sowohl eine maximale Reichweite als auch eine nachhaltige Nachfrage zu erzielen. Während in der Datenwissenschaftsabteilung parallel ein Modell zur Analyse von Reichweite entwickelt wird, konzentriert sich diese Arbeit speziell auf die Untersuchung des Nachfrageeffekts. Dadurch sollen Ergebnisse bereitgestellt werden, die als Referenz für die strategische Bewertung des Media-Mix dienen können. 
\subsection{Zielsetzung der Arbeit}
\label{ZielsetzungDerArbeit}
Das Ziel dieser Arbeit ist es, den Nachfrageeffekt verschiedener Media-Kanäle zu analysieren, sodass die Media-Abteilung fundierte Entscheidungen zur Verteilung der Werbeausgaben treffen kann. Dabei wird untersucht, wie sich die Mediaausgaben auf die Nachfrageänderung auswirken. Des Weiteren wird der \ac{ROAS} der verschiedenen Kanäle berechnet und die Verteilung der Nachfrageeffekte auf die einzelnen Media-Unterkanäle analysiert. Abschließend wird geprüft, ob die Ergebnisse des Marketing-Mix-Modells nachvollziehbar sind.\par
Die erste Problemstellung ist besonders herausfordernd, da Mediaausgaben in der Regel im Rahmen von Kampagnen getätigt werden. Dies bedeutet, dass an den meisten Tagen im Jahr keine Mediaausgaben verzeichnet werden, sondern nur während der Kampagnenperioden. 
\subsection{Methodik und Ansatz}
\label{MethodikUndAnsatz}
Im Rahmen dieser Bachelorarbeit werden mehrere methodische Schritte umgesetzt. Zunächst werden die Methoden in die Arbeit integriert, darunter sowohl die lineare als auch die multiple lineare Regression, die Methode der kleinsten Quadrate und die Huber-Loss-Funktion. Ihre Funktionsweisen werden ausführlich behandelt. Anschließend erfolgt eine deskriptive Analyse der vorbereiteten Daten, um die Mediaausgaben zu untersuchen. Dabei wird auch auf die Nachvollziehbarkeit der Daten, etwa die Verteilung der Mediaausgaben, eingegangen. Weiterhin wird die Wavelet-Funktion angewendet, um die für Media relevante Nachfrage herauszufiltern. Anschließend wird eine Huber-Regression auf die gefilterten Nachfragedaten angewandt. Zur Überprüfung der Nachvollziehbarkeit und Richtigkeit der Ergebnisse erfolgt ein Austausch mit der Fachabteilung.
\subsection{Aufbau der Arbeit}
Die vorliegende Bachelorarbeit ist in sechs Kapitel unterteilt, die einen systematischen Überblick über die Forschung zu dem Marketing-Mix-Modell (\ac{MMM}), der Integration der Methode der kleinsten Quadrate und der Huber-Loss-Funktion geben.\par 
Das erste Kapitel \nameref{Einführung} führt in die grundlegende Problemstellung ein und betont die Relevanz der Problemlösung für den Betrieb. In diesem Kapitel werden die Forschungsfragen erläutert. Zudem wird das Ziel der Arbeit sowie die nötigen Schritte dahin beleuchtet. \par
Das zweite Kapitel \nameref{GeschäftlicheGrundlagen} behandelt die Grundlagen des \ac{MMM} sowie des Marketings und erläutert den aktuellen Forschungsstand, um ein fundiertes fachliches Verständnis für die Modellierungsanforderungen zu schaffen. Dabei werden zentrale Marketing-Konzepte wie das Purchase-Funnel-Modell definiert und erklärt. Zudem fließen Erfahrungswerte anderer Unternehmen zur Kostenaufteilung von Media-Kanälen in die Analyse ein. Abschließend wird der aktuelle Stand des \ac{MMM} bei bonprix dargelegt und der aktuelle Stand der Media-Abteilung von bonprix wird beschrieben.\par 
Im dritten Kapitel \nameref{TheoretischeGrundlagen} wird das methodische Vorgehen beschrieben. Es wird erläutert, was eine deskriptive Analyse ist und wie die lineare sowie die multiple lineare Regression funktionieren. Anschließend werden die statistischen Grundlagen wie die Methode der kleinsten Quadrate, die Methode der kleinsten absoluten Abweichungen und die Huber-Loss-Funktion vorgestellt. Dazu werden auch Einschränkungen der Regression aufgelistet sowie die Anpassungsgüte und die Varianzanalyse vorgestellt. \par
Das vierte Kapitel \nameref{Umsetzung} dokumentiert die Arbeitsschritte. Zunächst erfolgt die Vorstellung der verwendeten Daten. Diese werden für den Einsatz verarbeitet und die deskriptive Analyse wird eingesetzt. Mit Hilfe der deskriptiven Analyse wird der Verlauf der Mediaausgaben und der daraus resultierenden Nachfrage untersucht und die Aufteilung der Mediaausgaben visualisiert. Nach der deskriptiven Analyse wird die Multikollinearität zwischen den Variablen untersucht und gegebenenfalls beseitigt. Somit werden Grundlagen für die Modellierung geschaffen. Anschließend werden die Variablen in das Modell eingesetzt. Die Ergebnisse werden analysiert und das Modell wird mit verschiedenen Kombinationen der Variablen getestet. Am Ende wird das Modell finalisiert. \par
Das fünfte Kapitel \nameref{schlussfolgerungenUndImplikationen} wertet die Modellergebnisse aus. Dabei werden die Nachfrageeffekte der verschiedenen Media-Kanäle untersucht und anhand der theoretischen Grundlagen aus Kapitel \nameref{TheoretischeGrundlagen} sowie der geschäftlichen Rahmenbedingungen aus Kapitel \nameref{GeschäftlicheGrundlagen} analysiert. Zudem werden verschiedene Werbewirkungen identifiziert und beschrieben. Zur Veranschaulichung der Nachfrage-Verteilung innerhalb der Media-Kanäle kommen Kreisdiagramme zum Einsatz. Abschließend werden die Einschränkungen des Modells erläutert sowie ein Ausblick und Handlungsempfehlungen gegeben. \par
Im sechsten Kapitel \nameref{Fazit} wird die Arbeit zusammengefasst. In diesem Kapitel wird die Forschungsfrage beantwortet. 
